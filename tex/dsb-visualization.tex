
\chapter{Using graphs and visualizing data}\label{ch:graphs}

\boxx{
	Required readings: 
	\itex{
		\item \textit{Using Graphs and Visualising Data} by Oliver \cite{Kirchkamp2018Using}: \url{https://www.kirchkamp.de/oekonometrie/pdf/gra-p.pdf}
		\item \citet[ch. 1]{Grolemund2018R}: \textit{Data Visualization with ggplot2}, See: \url{https://r4ds.had.co.nz/data-visualisation.html}
		\item Check out the fantastic webpage \url{https://www.data-to-viz.com/}
	}

	Recommended readings:
	\textit{The Visual Display of Quantitative Information} by Edward \cite{Tufte2001Visual}

	Required exercises:
	\itex{
		\item The \textit{Convergence} exercise found in \autoref{sec:convergence} and here: \url{https://github.com/hubchev/courses/blob/main/scr/convergence.R}
		\item The \textit{Regression analysis presentation} that can be found in the Appendix of these notes on pages \pageref{sec:regress_lecture}f and here: \url{https://htmlpreview.github.io/?https://github.com/hubchev/courses/blob/main/rmd/regress_lecture.html}
	}
}


\exex{What makes a graph ugly?}{
	Discuss what are the features of a good and bad graphical visualization of data. What do you think about the following graph?
	% TODO: \usepackage{graphicx} required
	\begin{center}
		\includegraphics[width=0.7\linewidth]{../../../pic/worst95}
	\end{center}
}




\subsection*{Matrix of Plots}
\begin{rblock1}
read_csv("https://raw.githubusercontent.com/hubchev/courses/main/dta/classdata.csv") %>%
	select(sex,weight,height,siblings) %>% 
	ggpairs(.,
	title = "Graphical Visualization of Our Survey in a Matrix: An Example", 
	mapping = ggplot2::aes(colour=sex), 
	lower = list(continuous = wrap("smooth", alpha = 0.9, size=1.1), 
	discrete = "blank", combo="count"), 
	diag = list(discrete="barDiag", 
	continuous = wrap("densityDiag", alpha=0.4 )), 
	upper = list(combo = wrap("box_no_facet", alpha=0.5),
	continuous = wrap("cor", size=4, align_percent=0.8))) + 
	theme(panel.grid.major = element_blank())    # remove gridlines 
\end{rblock1}

\begin{center}
\includegraphics[width=1.0\linewidth]{../../../pic/destat/classdata_matrix}
\end{center}

\itex{\item Can you interpret each plot in the matrix above?
\item Can you name the type of graph for each plot above?
}


\subsection*{Do's and don'ts of data visualization}

 Graphical representation is another way of analyzing numerical
data.
Data visualization can be a great way to get insights. It can help to communicate a large amount of information simply and intuitively. However, you should avoid a few all-too-common mistakes.

\begin{itemize}
\item Graphs can be misleading and that may hold even true for \textit{good} graphs sometimes in a way!
\item An overview of different types of graphs: \url{https://visme.co/blog/types-of-graphs/}
\item Nice animated graphs and the corresponding R code: \url{https://www.r-graph-gallery.com/animation.html}
\item Good graphs are easy to understand and eye catching.
\item Minimize colors and other attention-grabbing elements that are not directly related to the data of interest.
\item Show the full scale of the graph, then zoom to show the data of interest, if necessary. In other words, don't truncate the an axis or change the scaling within an axis just to make you your story more dramatic.
\item Label and describe your chart sufficiently so that everybody can fully understand the content of the shown data set and statistics, respectively.
\item For more tips, see: \url{https://guides.library.duke.edu/datavis/topten}
\end{itemize}



\boxb{\paragraph{Color blindness} Worldwide, there are approximately 300 million people with color blindness. Most colour blind people are able to see things as clearly as other people but they are unable to fully \textit{see} red, green or blue light. Thus, better rely on color schemes that are easy to \textit{see} for colorblind people.}


\paragraph{A note on pie chart} 
A pie chart shows a circle which is divided into
sectors that each represent a proportion of the whole.
\begin{center}
	\includegraphics[width=0.6\linewidth]{../../../pic/piebad}
	\note{Source: \url{https://commons.wikimedia.org/wiki/File:Piecharts.svg}}
\end{center}
Pie charts may look simple, but they’re tricky to get right.
The brain's not very good at comparing the size of angles and because there's no scale in pie plots, reading accurate values is difficult. Thus, pie charts are often poor at communicating data. They take up more space and are harder to read than alternatives. Unfortunately, people like to look at pies. 


\begin{center}
\includegraphics[width=1\textwidth,angle=90]{../../../pic/mas/chart_selection}
\end{center}





