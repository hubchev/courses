
\subsection{Price elasticity and market power}\label{price-elasticity-and-market-power} 

We know that the demand elasticity of price can be measured with:
\[ \frac{\triangle p/\bar{p}}{\triangle x / \bar{x}}. \]
Using differential calculus, the point-price elasticity of demand (PPD) can be written as:
\[ PPD = \frac{\partial p(x)}{\partial x} \cdot \frac{x}{p(x)} = p'(x) \frac{x}{p(x)} = x \frac{p'(x)}{p(x)} \]
Thus, if \(PPD = 0\), the price does not change if a single firm increases its quantity sold on the market.\footnote{Remember, $PPD$ is defined between 0 to $-\infty$, where 0 stands for perfect inelasticity (demand remains unchanged when prices increase) and $-\infty$ stands for perfect elasticity (demand goes down to zero if prices increase).} That means the firms are price takers, and their quantity sold has no impact on the price. If firms, however, have market power, \(PPD < 0\), which means that if a firm increases the quantity on the market, the price must fall. \textbf{The PPD can hence be interpreted as an indicator of the market power of firms.}

\subsection{Marginal revenue and price elasticity}\label{marginal-revenue-and-price-elasticity} 

Now, re-writing the MR function and substituting in PPD, we get:
\[ MR = p(x) + xp'(x) = p(x)\left(1 + \underbrace{x\frac{p'(x)}{p(x)}}_{PPD}\right)  = p(x)\left(1 + PPD\right). \]
This shows that MR is equal to zero when we have a unit demand elasticity, PPD, of \(-1\) and that MR is equal to the price if firms have no market power. Also see figure \ref{fig:ppd}. 

As shown earlier, in the monopoly output, marginal revenue and marginal cost are equal:
\begin{equation}
	MC = p(x) \cdot \left(1 + PPD\right)\label{eq:ppd}
\end{equation}
\begin{figure}
	\centering
	\includegraphics[width=0.75\textwidth]{fig/ppd.png}
	\caption{Price setting of a monopolist}\label{fig:ppd}
	\note{Graph stems from \citet[p.~344]{Anonymous2020Principles}}
\end{figure}

\subsection{Lerner index}\label{lerner-index}

The Lerner index is a measure of monopoly power, which equals the markup over marginal cost as a percentage of price. To obtain the Lerner index of monopoly power (or market power), let us rearrange \autoref{eq:ppd} as follows:

\[ \frac{MC - p(x)}{p(x)} = PPD = \text{Lerner Index} \]

If a firm does not have market power (\(PPD = 0\)), its price equals the marginal cost. When a firm's market power is high (up to \(|PPD| = \infty\)), the higher the markup that a firm sets. In perfect competition, since \(p\) and MC are equal, the Lerner Index is 0. A pure monopolist, on the other hand, can theoretically charge an infinite markup, which leads us to a Lerner index of 1.

\begin{center}\rule{0.5\linewidth}{0.5pt}\end{center}
