\chapter{Financial mathematics}\label{decision-making-and-money}


\section{Financial literacy}\label{financial-literacy}

You are financially literate if you understand and manage personal finances effectively. It involves having a basic understanding of financial concepts, such as budgeting, saving, investing, and managing debt. Financial literacy also includes knowledge of financial products and services, such as bank accounts, credit cards, loans, and insurance.
Being financially literate means having the skills and knowledge to make informed financial decisions, and being able to assess risks and opportunities when it comes to managing money. It is an important life skill that can help individuals achieve their financial goals, build wealth, and avoid financial pitfalls.

Being better-educated was always associated with having more financial knowledge (Figure
1) across the countries we examined,3 yet we also found that education is not enough. That
is, even well-educated people are not necessarily savvy about money.

Unfortunately, financial illiteracy is widespread. While being better-educated is associated with making better financial decisions on average, ``even well-educated people are not necessarily savvy about money'' \citep[p.~3]{Mitchell2015Financial}.

There are various attempts to assess the levels of financial literacy. See \url{https://www.oecd.org/finance/financial-education/measuringfinancialliteracy.htm} for example.

\exex{How to measure financial literacy}{
	One example of a comprehend way to measure financial literacy stems from the \emph{Standard \& Poor`s Ratings Services Global Financial Literacy Survey} \citep[see][]{Klapper2020Financial}. They ask the following multiple-choice questions:
	
	\begin{enumerate}		
		\item
		Suppose you have some money. Is it safer to put your money into one business or investment, or to put your money into multiple businesses or investments?
		
		\begin{enumerate}			
			\item
			One business or investment
			\item
			Multiple businesses or investments
			\item
			Don`t know
			\item
			Refused to answer
		\end{enumerate}
		\item
		Suppose over the next 10 years the prices of the things you buy double. If your income also doubles, will you be able to buy less than you can buy today, the same as you can buy today, or more than you can buy today?
		
		\begin{enumerate}			
			\item
			Less
			\item
			The same
			\item
			More
			\item
			Don`t know
			\item
			Refused to answer
		\end{enumerate}
		\item
		Suppose you need to borrow \$100. Which is the lower amount to pay back: \$105 or \$100 plus 3\%?
		
		\begin{enumerate}
			\item
			\$105
			\item
			\$100 plus 3\%
			\item
			Don`t know
			\item
			Refused to answer
		\end{enumerate}
		\item
		Suppose you put money in the bank for 2 years and the bank agrees to add 15\% per year to your account. Will the bank add more money to your account the second year than it did the first year, or will it add the same amount of money both years?
		
		\begin{enumerate}
			\item
			More
			\item
			The same
			\item
			Don`t know
			\item
			Refused to answer
		\end{enumerate}
		\item
		Suppose you had \$100 in a savings account and the bank adds 10\% per year to the account. How much money would you have in the account after 5 years if you did not remove any money from the account?
		
		\begin{enumerate}
			\item
			More than \$150
			\item
			Exactly \$150
			\item
			Less than \$150
			\item
			Don`t know
			\item
			Refused to answer
		\end{enumerate}
	\end{enumerate}
	
	These questions cover four fields of financial literacy, i.e., risk diversification, inflation and purchasing power, numeracy (simple calculations related to interest rates), and compound interest (interest payments increase exponentially over time). Knowledge in these concepts is important to make good financial decisions and to manage risk.
	
	Try to answer these quesions and compare your performance with the results shown in \citet{Klapper2020Financial}.
	
}

\exex{The big three questions}{
	Referring to \citet{Mitchell2015Financial}, only 21.7\% of individuals in Germany with a lower secondary education and 72\% of those with tertiary education can correctly answer all three of the following questions. Try it yourself!
	
	\begin{enumerate}
		\def\labelenumi{\arabic{enumi}.}
		
		\item
		Suppose you had \$100 in a savings account and the interest rate was 2\% per year. After 5 years, how much do you think you would have in the account if you left the money to grow?
		
		\begin{enumerate}
			\def\labelenumii{\alph{enumii})}
			
			\item
			More than \$102
			\item
			Exactly \$102
			\item
			Less than \$102
			\item
			Do not know
			\item
			Refuse to answer
		\end{enumerate}
		\item
		Imagine that the interest rate on your savings account was 1\% per year and inflation was 2\% per year. After 1 year, how much would you be able to buy with the money in this account?
		
		\begin{enumerate}
			\def\labelenumii{\alph{enumii})}
			
			\item
			More than today
			\item
			Exactly the same
			\item
			Less than today
			\item
			Do not know
			\item
			Refuse to answer
		\end{enumerate}
		\item
		Please tell me whether this statement is true or false: \emph{``Buying a single company's stock usually provides a safer return than a stock mutual fund.''}
		
		\begin{enumerate}
			\def\labelenumii{\alph{enumii})}
			
			\item
			True
			\item
			False
			\item
			Do not know
			\item
			Refuse to answer
		\end{enumerate}
	\end{enumerate}
	
	These three questions are designed to measure
	\citet{Lusardi2014Economic} reports that in many countries the financial illiteracy is considerably high as the following table shows:
	
	\begin{longtable}[]{@{}lcc@{}}
		\caption{Financial literacy around the World}\tabularnewline
		\toprule\noalign{}
		& \% all correct & \% none correct \\
		\midrule\noalign{}
		\endfirsthead
		\toprule\noalign{}
		& \% all correct & \% none correct \\
		\midrule\noalign{}
		\endhead
		\bottomrule\noalign{}
		\endlastfoot
		Germany & 57 & 10 \\
		Netherlands & 46 & 11 \\
		United States & 35 & 10 \\
		Italy & 28 & 20 \\
		Sweden & 27 & 11 \\
		Japan & 27 & 17 \\
		New Zealand & 27 & 4 \\
		Russia & 3 & 28 \\
	\end{longtable}
	
}

\section{Common investment mistakes}\label{common-investment-mistakes}

Investing can be a daunting task, but avoiding some common investment mistakes can help set you on the right path to financial success.
The following list list shows according to \citet{Stammers2016Tips} the \emph{Top 20 common investment mistakes} without the explanations provided in the paper:

\begin{itemize}
	
	\item
	Expecting too much or using someone else's expectations
	\item
	Not having clear investment goals
	\item
	Failing to diversify enough
	\item
	Focusing on the wrong kind of performance
	\item
	Buying high and selling low
	\item
	Trading too much and too often
	\item
	Paying too much in fees and commissions
	\item
	Focusing too much on taxes
	\item
	Not reviewing investments regularly
	\item
	Taking too much, too little, or the wrong risk
	\item
	Not knowing the true performance of your investments
	\item
	Reacting to the media
	\item
	Chasing yield
	\item
	Trying to be a market timing genius
	\item
	Not doing due diligence
	\item
	Working with the wrong adviser
	\item
	Letting emotions get in the way
	\item
	Forgetting about inflation
	\item
	Neglecting to start or continue
	\item
	Not controlling what you can
\end{itemize}

\exex{Common investment mistakes}{
	Read the paper and summarize the 20 mistakes.
}

\section{Simple financial mathematics}\label{simple-financial-mathematics}

I discuss financial mathematics in the following chapter just briefly. If you want to gather a deeper understanding, I recommend the open textbook of \citet{Dahlquist2022Principles} or the respective chapters of \citet{Wilkinson2022Managerial}.


\subsection{Simple Interest}\label{simple-interest}

Suppose \(r\) denotes annual interest rates, \(P\) denotes the initial deposit which earns the interest, \(A\) denotes the value of the deposit at the end of an investment. Then, the relationship of these for a single year is

\[
A=P+Pr=P(1+r)
\]

and for many years, \(t\), it is

\[
A=P(1+rt)
\]

which is the simple interest formula. It gives the amount due when the annual interests does not become part of the deposit \(P\).

\subsection{Compound interest}\label{compound-interest}

If the annual interest, \(P(1+r)\), is added to \(P\), we need a formula that takes this into account, and for two periods this is

\[
A=P\cdot  [(1+r)\cdot(1+r)]=P(1+r)^2
\]

and for t periods

\[
A=P(1+r)^t.
\]

Compound interest is the addition of interest to the principal sum of a loan or deposit, or in other words, interest on principal plus interest. It is the result of reinvesting interest, or adding it to the loaned capital rather than paying it out, or requiring payment from borrower, so that interest in the next period is then earned on the principal sum plus previously accumulated interest.

\[
A=P\left(1+\frac{r}{n}\right)^{nt}
\]


\subsubsection*{Example}\label{example}

Suppose a principal amount of \$1,500 is deposited in a bank paying an annual interest rate of 4.3\%, compounded quarterly. Then the balance after 6 years is found by using the formula above, with \(P = 1500\), \(r = 0.043\) (4.3\%), \(n = 4\), and \(t = 6\):

\[
A=1500\times\left(1+\frac{0.043}{4}\right)^{4\times 6}\approx 1938.84
\]

So the amount \(A\) after 6 years is approximately \$1,938.84.

Subtracting the original principal from this amount gives the amount of interest received: \(1938.84-1500=438.84\)

\subsection{Continuously compounded interest}\label{continuously-compounded-interest}

As \(n\), the number of compounding periods per year, increases without limit, the case is known as continuous compounding, in which case the effective annual rate approaches an upper limit of \(e^r- 1\), where \(e\) is a mathematical constant that is the base of the natural logarithm.

Continuous compounding can be thought of as making the compounding period infinitesimally small, achieved by taking the limit as \(n\) goes to infinity. The amount after \(t\) periods of continuous compounding can be expressed in terms of the initial amount \(P\) as

\[
A=Pe^{rt}
\]

\subsection{Present value}\label{present-value}

The present is the value of an expected income stream determined as of the date of valuation. The present value is usually less than the future value because money has interest-earning potential, a characteristic referred to as the time value of money, except during times of zero- or negative interest rates, when the present value will be equal or more than the future value. Time value can be described with the simplified phrase, ``A dollar today is worth more than a dollar tomorrow'`. Here, 'worth more' means that its value is greater than tomorrow. A dollar today is worth more than a dollar tomorrow because the dollar can be invested and earn a day's worth of interest, making the total accumulate to a value more than a dollar by tomorrow.

\[
P=Ae^{-rt}
\]

\section{Net present value and internal rate of return}\label{net-present-value-and-internal-rate-of-return}

When making decisions about financial products such as investments or loans, it is important to consider their long-term impact on your finances. \emph{Net Present Value} (NPV) and \emph{Internal Rate of Return} (IRR) are two key indicators that can help guide decision making and determine whether a financial product is a good investment.

Net Present Value (NPV) is the difference between the present value of all cash inflows and the present value of all cash outflows over a given time period. The formula to calculate NPV is:

\[ NPV = \sum_{n=1}^{N} \frac{C_n}{(1+r)^n} - C_0 \]

where \(C_n\) denotes net cash inflow during the period \(n\), \(r\) the discount rate, or the cost of capital, \(n\) the number of periods, and \(C_0\) the initial investment.

In other words, NPV helps determine the current value of future cash flows, adjusted for the time value of money. A positive NPV indicates that an investment is expected to generate a return greater than the cost of capital, while a negative NPV suggests that the investment is likely to result in a loss.

Internal Rate of Return (IRR), on the other hand, is the discount rate that makes the NPV of all cash inflows equal to the NPV of all cash outflows. The formula to calculate IRR is:

\[ 0 = \sum_{n=0}^{N} \frac{C_n}{(1+IRR)^n}  \]

where \(C_n\) denotes the net cash inflow during the period \(n\), \(IRR\) the internal rate of return, \(n\) the number of periods, and \(C_0\) the initial investment.

IRR can be thought of as the rate of return an investment generates over time, taking into account the time value of money. When comparing different investment opportunities, a higher IRR generally indicates a more profitable investment.

Both NPV and IRR are important tools to help individuals make informed decisions about financial products. By comparing the NPV and IRR of different investment options, individuals can determine which investments are likely to generate the greatest returns over time, and which products may not be worth the initial investment.

It is worth noting that while NPV and IRR are useful indicators for decision making, they are not the only factors to consider. Individuals should also consider other important factors such as risk, liquidity, and diversification when evaluating different financial products. By taking a holistic approach and considering all relevant factors, individuals can make informed decisions that are best suited to their financial goals and circumstances.

\exex{Investment case}{
	
	You deposit 1,000 euros today into a savings account with an annual interest rate of 5\% for 2 years. What is the balance after 2 years with annual, semi-annual (4 interest payments per year), and continuous compounding?
	
}

\exex{Present value}{
	
	You want to have 100,000 in 10 years, and you can save money with an interest rate of 5\% p.a. How much do you need to invest today for annual, semi-annual (4 interest periods), and continuous compounding to achieve your goal?
	
}

\exex{Invest in A or B}{
	You are considering investing in project A or B.
	
	\textbf{Project A:} It costs 50,000 today and is expected to generate cash flows of 20,000 per year for the next 5 years. You have a required rate of return of 8\%.
	
	\textbf{Project B:} It costs 50,000 today and you get 100,000 back in 5 years.
	
	Calculate the value of your invest after five years. Which investment is the better one?
	
	
}

\exex{Net present value}{
	You are considering investing in project A or B.
	
	\textbf{Project A:} It costs 50,000 today and is expected to generate cash flows of 20,000 per year for the next 5 years. You have a required rate of return of 8\%.
	
	\textbf{Project B:} It costs 50,000 today and you get 100,000 back in 5 years.
	
	Calculate the net present value of both projects and decide where to invest.
	
	
}

\exex{Internal rate of return}{
	You are considering investing in project A or B.
	
	\textbf{Project A:} It costs 50,000 today and is expected to generate cash flows of 20,000 per year for the next 5 years. You have a required rate of return of 8\%.
	
	\textbf{Project B:} It costs 50,000 today and you get 100,000 back in 5 years.
	
	Calculate the internal rate of return of both projects with the help of a software package such as \emph{Excel} or \emph{Libre Calc} and decide where to invest.
	
	
}

\exex{Rule of 70}{
	The \emph{Rule of 70} is often used to approximate the time required for a growing series to double. To understand this rule calculate how many periods it takes to double your money when it growth at a constant rate of 1\% each period.
	
	
}

\section{Logarithmic and exponential function}\label{logarithmic-and-exponential-function}

\begin{figure}
\centering
\includegraphics[width=0.3\textwidth]{fig/logexp1.png}
\caption{\label{fig:logexp1} Logarithmic and exponential function}
\end{figure}

\begin{figure}
\centering
\includegraphics[width=0.3\textwidth]{fig/logexp2.png}
\caption{\label{fig:logexp2} Logarithmic and exponential function zoomed in}
\end{figure}

\subsection{Logarithmic function}\label{logarithmic-function}

Maybe you have heard about the logarithm, and I'm quite sure you know the `log' button on your calculator. If you wonder what it actually is and why it is so important for calculating with growth rates, this section is for you.

Consider the following equations and then explain to me what the logarithm is:

\begin{align*}
2\cdot2\cdot2\cdot2 &= 16 \\
2^4 &= 16 \\
2 &= 16^{\frac{1}{4}} \\
\log_2 16 &= 4 
\end{align*}

Now, let us abstract from that concrete example and generalize things a bit:

\begin{align*}
x^n &= y \\
\log_x y &= n \\
\log x^n &= \log y \\
n\cdot \log x &= \log y \\
n &= \frac{\log y}{\log x} 
\end{align*}

Here are some more examples:

\begin{align*}
4 &= \frac{\log 16}{\log 2} \\
\log_{10}16 &= 1.20411998265592 \\
\log_{10}2 &= 0.301029995663981 \\
\log 16 &= 1.20411998265592 \\
\log 2 &= 0.301029995663981 \\
\end{align*}

\exex{Calculate with log}{

Calculate a logarithmic function without a calculator. Hint: The result is an integer.

\begin{itemize}
	\item
	\(\log_2 16 = ?\)
	\item
	\(\log_3 243 = ?\)
	\item
	\(\log_5 125 = ?\)
	\item
	\(\log_3 81 = ?\)
	\item
	\(\log_2 \left(\frac{1}{8}\right)\)
\end{itemize}
}

\subsection{Exponential function}\label{exponential-function}
\subsubsection{Definition}\label{definition}

Let us consider the function \(f(x) = 2^{x}\) in table Table 1 and figure \ref{fig:IntroExpLogs-1}.

\begin{longtable}[]{@{}lll@{}}
\caption{Table 1: The exponential function \(2^{x}\) in a table}\tabularnewline
\toprule\noalign{}
\(x\) & \(f(x)\) & \((x,f(x))\) \\
\midrule\noalign{}
\endfirsthead
\toprule\noalign{}
\(x\) & \(f(x)\) & \((x,f(x))\) \\
\midrule\noalign{}
\endhead
\bottomrule\noalign{}
\endlastfoot
\(-3\) & \(2^{-3} = \frac{1}{8}\) & \((-3, \frac{1}{8})\) \\
\(-2\) & \(2^{-2} = \frac{1}{4}\) & \((-2, \frac{1}{4})\) \\
\(-1\) & \(2^{-1} = \frac{1}{2}\) & \((-1, \frac{1}{2})\) \\
\(0\) & \(2^{0} = 1\) & \((0 ,1)\) \\
\(1\) & \(2^{1} = 2\) & \((1, 2)\) \\
\(2\) & \(2^{2} = 4\) & \((2,4)\) \\
\(3\) & \(2^{3} = 8\) & \((3, 8)\) \\
\end{longtable}

\begin{figure}
\centering
\includegraphics[width=0.5\textwidth]{fig/IntroExpLogs-1.png}
\caption{\label{fig:IntroExpLogs-1} The exponential function \(2^{x}\) visualized}
\end{figure}

A function of the form \(f(x) = b^{x}\) where \(b\) is a fixed real number, \(b > 0\), \(b \neq 1\) is called a \textbf{base \(b\) exponential function}.

\begin{itemize}
\item
Therefore, \(b\) is the factor by which \(f(x)\) increases or decreases when \(x\) increases by one unit.
\item
For \(b > 1\), the function \(f(x)\) is strictly increasing.
\item
For \(0 < b < 1\), the function \(f(x)\) is strictly decreasing.
\end{itemize}

\subsubsection{\texorpdfstring{The number \(e\)}{The number e}}\label{the-number-e}

The most important base for exponential functions is the irrational number

\(e \approx 2.71828182845904523536028747135266249775724709369995\)

It is sometimes called \emph{Euler's number}, after the Swiss mathematician Leonhard Euler (1707-1783).
It can be expressed as:

\begin{itemize}
\item
\(e = \sum \limits _{n=0}^{\infty }{\frac {1}{n!}} = 1+{\frac {1}{1}}+{\frac {1}{1\cdot 2}}+{\frac {1}{1\cdot 2\cdot 3}}+\cdots\)
\item
\(e^{x} = 1+{x \over 1!}+{x^{2} \over 2!}+{x^{3} \over 3!}+\cdots = \sum _{n=0}^{\infty }{\frac {x^{n}}{n!}}\)
\item
\(e^k = \lim _{n\to \infty }\left(1+{\frac {k}{n}}\right)^{n}\)
\end{itemize}

Watch the YouTube clips \href{https://youtu.be/_-x90wGBD8U}{Logarithms - What is e? \textbar{} Euler's Number Explained \textbar{} Don't Memorise} by \emph{Infinity Learn Class 9\&10} (see figure \ref{fig:whatse}) and \href{https://youtu.be/m2MIpDrF7Es}{What's so special about Euler's number e? \textbar{} Chapter 5, Essence of calculus} by \emph{3Blue1Brown} (see figure \ref{fig:euler}).

\begin{figure}
\centering
\includegraphics[width=0.5\textwidth]{fig/whatse.png}
\caption{\label{fig:whatse} Euler's number explained}
\end{figure}

\begin{figure}
\centering
\includegraphics[width=0.5\textwidth]{fig/euler.png}
\caption{\label{fig:euler} What's so special about Euler's number e?}
\end{figure}





























\section{A note on growth rates and the logarithm}\label{a-note-on-growth-rates-and-the-logarithm}

Most data are recorded for discrete periods of time (e.g., quarters, years). Consequently, it is often useful to model economic dynamics in discrete periods of time. A good linear approximation to a growth rate from time \(t=0\) to \(t=1\) in \(x\) is \(\ln x_0 - \ln x_1\):

\[
\frac{x_1 - x_0}{x_0} \approx \ln x_1 - \ln x_0
\]

Let us prove that with some numbers of per capita real GDP for the US and Japan in 1950 and 1989:

\begin{center}
	\begin{tabular}{ccc}\toprule
		1950 & 1950 & 1989 \\ \midrule
		US & 8611 & 18317 \\ 
		Japan & 1563 & 15101 \\ \bottomrule
	\end{tabular} 
\end{center}

What are the annual average growth rates over this period for the US and Japan? Here is one way to answer this question:

\[
Y_{1989} = (1 + g)^{39} \cdot Y_{1950}
\]

Consequently, \(g\) can be calculated as:

\[
(1 + g) = \left(\frac{Y_{1989}}{Y_{1950}}\right)^{\frac{1}{39}}
\]

Yielding \(g = 0.0195\) for the US and \(g = 0.0597\) for Japan. The US grew at an average growth rate of about 2\% annually over the period, while Japan grew at about 6\% annually.

The following method gives a close approximation to the answer above and will be useful in other contexts. A useful approximation is that for any small number \(x\): \(\ln (1 + x) \approx x\)

Now, we can take the natural log of both sides of:

\[
\frac{Y_{1989}}{Y_{1950}} = (1 + g)^{39}
\]

to get:

\[
\ln (Y_{1989}) - \ln (Y_{1950}) = 39 \cdot \ln (1 + g)
\]

which rearranges to:

\[
\ln (1 + g) = \frac{\ln (Y_{1989}) - \ln (Y_{1950})}{39}
\]

and using our approximation:

\[
g \approx \frac{\ln (Y_{1989}) - \ln (Y_{1950})}{39}
\]

In other words, log growth rates are good approximations for percentage growth rates. Calculating log growth rates for the data above, we get \(g \approx 0.0194\) for the US and \(g \approx 0.0582\) for Japan. The approximation is close for both.

\subsection*{Plotting growth using the logarithm}\label{plotting-growth-using-the-logarithm}

Recall that, with a constant growth rate \(g\) and starting from time 0, output in time \(t\) is:

\[Y_t = (1 + g)^t \cdot Y_0\]

Taking natural logs of both sides, we have:

\[\ln Y_t = \ln Y_0 + t \cdot \ln (1 + g)\]

We see that log output is linear in time. Thus, if the growth rate is constant, a plot of log output against time will yield a straight line. Consequently, plotting log output against time is a quick way to \textbf{eyeball} whether growth rates have changed over time.

\begin{figure}
	\centering
	\includegraphics[width=0.5\textwidth]{fig/logplot.png}
	\caption{\label{fig:logplot} Log Plot}
\end{figure}

In figure \ref{fig:logplot} and \ref{fig:loginscale} you see a semi-logarithmic plot that has one axis on a logarithmic scale and the other on a linear scale. It is useful for data with exponential relationships, where one variable covers a large range of values, or to zoom in and visualize that what seems to be a straight line in the beginning is, in fact, the slow start of a logarithmic curve that is about to spike, and changes are much bigger than thought initially.

\begin{figure}
	\centering
	\includegraphics[width=0.5\textwidth]{fig/LogLinScale.png}
	\caption{\label{fig:loginscale} Log-Lin Scale}
\end{figure}


\exex{Investments over time}{
	
	Describe the formulas to describe the growth process of an investment over time when time is discrete and when time is continuous.
	
	
}

\exex{Exponential growth}{
	
	Sketch a timeline for each of the following series:
	
	\begin{itemize}
		
		\item
		\(a_t=a_{t-1}+g\)
		\item
		\(\ln(a_t)\)
		\item
		\(b_t=b_{t-1}\cdot (1+g)\)
		\item
		\(\ln b_t\)
	\end{itemize}
	
}

\exex{COVID and how to plot it}{
	
	I downloaded the complete \textbf{Our World in Data COVID-19} dataset from \href{https://ourworldindata.org/coronavirus/country/germany}{ourworldindata.org}. I created some graphs which I will show you below. Can you discuss the scaling and how to interpret them? What is your opinion on these graphs? Are some of them a bit misleading (at least if you don't look twice)?	
}

\begin{figure}
	\centering
	\includegraphics[width=0.5\textwidth]{fig/total.png}
	\caption{Total Cases}
\end{figure}

\begin{figure}
	\centering
	\includegraphics[width=0.5\textwidth]{fig/total2.png}
	\caption{Total Cases 2}
\end{figure}

\begin{figure}
	\centering
	\includegraphics[width=0.5\textwidth]{fig/total3.png}
	\caption{Total Cases 3}
\end{figure}

\begin{figure}
	\centering
	\includegraphics[width=0.5\textwidth]{fig/new.png}
	\caption{New Cases}
\end{figure}

\begin{figure}
	\centering
	\includegraphics[width=0.5\textwidth]{fig/new2.png}
	\caption{New Cases 2}
\end{figure}

\begin{figure}
	\centering
	\includegraphics[width=0.5\textwidth]{fig/new3.png}
	\caption{New Cases 3}
\end{figure}


\section{Solutions}



\solx{How to measure financial literacy}{

Correct answers are: 1b, 2b, 3b, 4a, 5a.
}

\solx{The big three questions}{
Correct answers are:

1a, 2c, 3b
}

\solx{Common investment mistakes}{

\begin{itemize}

	\item
	\textbf{Expecting too much or using someone else's expectations:} Nobody can tell you what a reasonable rate of return is without having an understanding of you, your goals, and your current asset allocation.
	\item
	\textbf{Not having clear investment goals:} Too many investors focus on the latest investment fad or on maximizing short-term investment return instead of designing an investment portfolio that has a high probability of achieving their long-term investment objectives.
	\item
	\textbf{Failing to diversify enough:} The best course of action is to find a balance. Seek the advice of a professional adviser.
	\item
	\textbf{Focusing on the wrong kind of performance:} If you find yourself looking short term, refocus.
	\item
	\textbf{Buying high and selling low:} Instead of rational decision making, many investment decisions are motivated by fear or greed.
	\item
	\textbf{Trading too much and too often:} You should always be sure you are on track. Use the impulse to reconfigure your investment portfolio as a prompt to learn more about the assets you hold instead of as a push to trade.
	\item
	\textbf{Paying too much in fees and commissions:} Look for funds that have fees that make sense and make sure you are receiving value for the advisory fees you are paying.
	\item
	\textbf{Focusing too much on taxes:} It is important that the impetus to buy or sell a security is driven by its merits, not its tax consequences.
	\item
	\textbf{Not reviewing investments regularly:} Check in regularly to make sure that your investments still make sense for your situation and that your portfolio doesn't need rebalancing.
	\item
	\textbf{Taking too much, too little, or the wrong risk:} Make sure that you know your financial and emotional ability to take risks and recognize the investment risks you are taking.
	\item
	\textbf{Not knowing the true performance of your investments:}
	Many investors do not know how their investments have performed in the context of their portfolio. You must relate the performance of your overall portfolio to your plan to see if you are on track after accounting for costs and inflation.
	\item
	\textbf{Reacting to the media:}
	Using the news channels as the sole source of investment analysis is a common investor mistake. Successful investors gather information from several independent sources and conduct their own proprietary research and analysis.
	\item
	\textbf{Chasing yield:}
	High-yielding assets can be seductive, but the highest yields carry the highest risks. Past returns are no indication of future performance. Focus on the whole picture and don't get distracted while disregarding risk management.
	\item
	\textbf{Trying to be a market timing genius:}
	Market timing is very difficult and attempting to make a well-timed call can be an investor's undoing. Consistently contributing to your investment portfolio is often better than trying to trade in and out in an attempt to time the market.
	\item
	\textbf{Not doing due diligence:}
	Check the training, experience, and ethical standing of the people managing your money. Ask for references and check their work on the investments they recommend. Taking the time to do due diligence can help avoid fraudulent schemes and provide peace of mind.
	\item
	\textbf{Working with the wrong adviser:}
	An investment adviser should share a similar philosophy about investing and life in general. The benefits of taking extra time to find the right adviser far outweigh the comfort of making a quick decision.
	\item
	\textbf{Letting emotions get in the way:}
	Investing can bring up significant emotional issues that can impede decision-making. A good adviser can help construct a plan that works no matter what the answers to important financial questions are.
	\item
	\textbf{Forgetting about inflation:}
	It's important to focus on real returns after accounting for fees and inflation. Even if the economy is not in a massive inflationary period, some costs will still rise, so it's important to focus on what you can buy with your assets, rather than their value in dollar terms.
	\item
	\textbf{Neglecting to start or continue:}
	Investment management requires continual effort and analysis to be successful. It's important to start investing and continue to invest over time, even if you lack basic knowledge or have experienced investment losses.
	\item
	\textbf{Not controlling what you can:}
	While you can't control what the market will bear, you can control how much money you save. Continually investing capital over time can have as much influence on wealth accumulation as the return on investment and increase the probability of reaching your financial goals.
\end{itemize}
}

\solx{Investment case}{
\begin{itemize}
	\item
	Annual compounding:
\end{itemize}

\[1,000 \text{€} \cdot (1 + 0.05) ^ 2 = 1,102.50 \text{€}\]

\begin{itemize}
	\item
	Semi-annual compounding:
\end{itemize}

\[ 1,000 \text{€} \cdot \left( 1 +  \frac{0.05}{4} \right)^{2 \cdot 4} = 1,104.49 \text{€}\]
- Continuous compounding:

\[1,000 \text{€} \cdot e^{0.05 \cdot 2} = 1,105.17 \text{€}\]
}

\solx{Present value}{
	
\begin{itemize}
	\item
	Annual compounding:
\end{itemize}

The formula for the future value of a present amount with annual compounding is:
\[ 
V_{\text{future}} = V_{\text{present}} \cdot (1 + i)^t 
\]

To calculate the present value, we need to rearrange the above formula for Present Value:
\[
V_{\text{present}} = \frac{V_{\text{future}}}{(1 + i)^t}
\]
\[
V_{\text{present}} = \frac{100,000}{(1 + 0.05)^{10}} \approx 61,391
\]

\begin{itemize}
	\item
	Semi-annual compounding (4 interest periods per year):
\end{itemize}

The formula for the future value of a present amount with semi-annual compounding is:
\[
V_{\text{future}} = V_{\text{present}} \cdot \left(1 + \frac{i}{p}\right)^{p\cdot t}.
\]
To calculate the present value, we need to rearrange the above formula for Present Value:
\[
V_{\text{present}} = \frac{V_{\text{present}}}{\left(1 +\frac{i}{p}\right)^{p\cdot t}}
\]
\[
\frac{100,000}{\left(1 + \frac{0.05}{4}\right)^{4\cdot 10}} \approx 60,841
\]

\begin{itemize}
	\item
	Continuous compounding:
\end{itemize}

The formula for the future value of a present amount with continuous compounding is:
\[
V_{\text{future}} = V_{\text{present}} \cdot e^{i \cdot t}
\]
To calculate the present value, we need to rearrange the above formula for present value:
\[
V_{\text{present}}  = \frac{V_{\text{future}}}{e^{i\cdot t}}
\]
\[
V_{\text{present}} = \frac{100,000}{e^{0.05 \cdot 10}} \approx 60,653
\]
}

\solx{Invest in A or B}{

\[
V_{A}^{t=5}\approx 117,332
\]
\[
V_{B}^{t=5}= 100,000
\]

Thus, we should prefer project A.
}

\solx{Net present value}{
Assuming that the cash flows occur at the end of each year, we can use the following formula to calculate the NPV of the project:

\[ 
NPV = \sum_{n=1}^{N} \frac{C_n}{(1+r)^n} - C_0 
\]

\[
NPV_A = -50,000 + \frac{20,000}{(1 + 0.08)^1} + 
\frac{20,000 }{ (1 + 0.08)^2} + 
\frac{20,000 }{ (1 + 0.08)^3} + 
\frac{20,000 }{ (1 + 0.08)^4} + 
\frac{20,000 }{ (1 + 0.08)^5}
\]

\[
NPV_A = -50,000 + 18,518.52 + 17,146.77 + 15,876.64 + 14,700.59 + 13,611.66 \approx 29,854
\]

\[
NPV_B = -50,000 + \frac{100,000 }{ (1 + 0.08)^5} = -50,000+68058,31\approx 18058
\]
Since the \(NPV_A>NPV_B\), we should invest in project A.
}

\solx{Internal rate of return}{
Assuming that the cash flows occur at the end of each year, we can use the following formula to calculate the IRR of the project:
\[ 0 = \sum_{n=0}^{N} \frac{C_n}{(1+IRR)^n}  \]
\[
0 = -50,000 + 
\frac{20,000}{(1 + IRR)^1} 
+ \frac{20,000 }{ (1 + IRR)^2 }
+ \frac{20,000 }{ (1 + IRR)^3 }
+ \frac{20,000 }{ (1 + IRR)^4 }
+ \frac{20,000 }{ (1 + IRR)^5 }
\]

Solving for \(IRR\) is not that easy. Using a spreadsheet program, we get \(IRR_A\approx 28.68%
\) and \(IRR_B\approx 14.87%
\). Thus, project A seems to be better.
}

\solx{Rule of 70}{
What is the time required for a growing variable to double?

Let \(X\) be the initial value of a growing variable, and \(Y\) denote the terminal value at time \(t + n\). The relationship between the two is given by

\[Y = X(1+g)^n\]

where \(g\) is the annual growth rate. As we are interested in the time span required for \(X\) to double, \(Y = 2\), and

\[2 = (1+g)^n\]

Taking natural logarithms (logarithm to the base of \(e\)), we get

\[\ln 2 = n \ln (1+g)\]

and hence

\[n = \frac{\ln 2}{\ln (1+g)} \quad (*).\]

This is the exact number of time periods required for a growing variable to double its size.

One can approximate \(n\) using the definition of \(e^x\):

\[e^x = 1 + x + \frac{x^2}{2!} + \frac{x^3}{3!} + \ldots + \frac{x^n}{n!} + R_n\]

where the remainder term \(R_n \rightarrow 0\) as \(n \rightarrow \infty\). Ignoring high-order terms, for small \(x\), it may be approximated by

\[e^x \approx 1 + x\]

Taking logarithms of both sides, we get

\[x \approx \ln (1 + x) \quad (**).\]

Using (**), equation (*) may be approximated as

\[n \approx \frac{\ln 2}{g} = \frac{0.693147}{g} \approx \frac{70}{g\%}\]

This is the origin of the Rule of 70.

Using the number \(e\) right away is simpler:

\begin{align*}
	(e^r)^t &= 2 \\
	\ln e^{rt} &= \ln 2 \\
	rt &= \ln 2 \\
	t &= \frac{\ln 2}{r} \\
	t &\approx \frac{0.693147}{r}
\end{align*}
}


\solx{Calculate with log}{

\begin{itemize}
	\item
	\(\log_2 16=4\) because \(2^4=16\)
	\item
	\(\log_3 243=5\) because \(3^5=243\)
	\item
	\(\log_5 125=3\) because \(5^3=125\)
	\item
	\(\log_3 81=4\) because \(3^4=81\)
	\item
	\(\log_2 \left(\frac{1}{8}\right)=-3\) because \(2^{-3}=\frac{1}{8}\)
\end{itemize}
}

\solx{Investments over time}{
The formula under discrete time is:
\[
Y_t=Y_0\cdot (1+g)^t
\]
The formula under continuous time is:
\[
Y_t=Y_0\cdot e^{gt}
\]
}

\solx{Exponential growth}{
\autoref{fig:expgrowth_graphs} provides the solution.
}

\begin{figure}
	\centering
	\includegraphics[width=0.75\textwidth]{fig/expgrowth_graphs.png}
	\caption{Various growth functions}\label{fig:expgrowth_graphs}
\end{figure}
