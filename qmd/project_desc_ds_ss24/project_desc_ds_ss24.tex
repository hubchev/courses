% Options for packages loaded elsewhere
\PassOptionsToPackage{unicode}{hyperref}
\PassOptionsToPackage{hyphens}{url}
\PassOptionsToPackage{dvipsnames,svgnames,x11names}{xcolor}
%
\documentclass[
  a4paper,
  onecolumn,
  oneside]{scrartcl}

\usepackage{amsmath,amssymb}
\usepackage{iftex}
\ifPDFTeX
  \usepackage[T1]{fontenc}
  \usepackage[utf8]{inputenc}
  \usepackage{textcomp} % provide euro and other symbols
\else % if luatex or xetex
  \usepackage{unicode-math}
  \defaultfontfeatures{Scale=MatchLowercase}
  \defaultfontfeatures[\rmfamily]{Ligatures=TeX,Scale=1}
\fi
\usepackage{lmodern}
\ifPDFTeX\else  
    % xetex/luatex font selection
\fi
% Use upquote if available, for straight quotes in verbatim environments
\IfFileExists{upquote.sty}{\usepackage{upquote}}{}
\IfFileExists{microtype.sty}{% use microtype if available
  \usepackage[]{microtype}
  \UseMicrotypeSet[protrusion]{basicmath} % disable protrusion for tt fonts
}{}
\makeatletter
\@ifundefined{KOMAClassName}{% if non-KOMA class
  \IfFileExists{parskip.sty}{%
    \usepackage{parskip}
  }{% else
    \setlength{\parindent}{0pt}
    \setlength{\parskip}{6pt plus 2pt minus 1pt}}
}{% if KOMA class
  \KOMAoptions{parskip=half}}
\makeatother
\usepackage{xcolor}
\usepackage[lmargin=1in,rmargin=1in,tmargin=1in,bmargin=1in]{geometry}
\setlength{\emergencystretch}{3em} % prevent overfull lines
\setcounter{secnumdepth}{5}
% Make \paragraph and \subparagraph free-standing
\ifx\paragraph\undefined\else
  \let\oldparagraph\paragraph
  \renewcommand{\paragraph}[1]{\oldparagraph{#1}\mbox{}}
\fi
\ifx\subparagraph\undefined\else
  \let\oldsubparagraph\subparagraph
  \renewcommand{\subparagraph}[1]{\oldsubparagraph{#1}\mbox{}}
\fi


\providecommand{\tightlist}{%
  \setlength{\itemsep}{0pt}\setlength{\parskip}{0pt}}\usepackage{longtable,booktabs,array}
\usepackage{calc} % for calculating minipage widths
% Correct order of tables after \paragraph or \subparagraph
\usepackage{etoolbox}
\makeatletter
\patchcmd\longtable{\par}{\if@noskipsec\mbox{}\fi\par}{}{}
\makeatother
% Allow footnotes in longtable head/foot
\IfFileExists{footnotehyper.sty}{\usepackage{footnotehyper}}{\usepackage{footnote}}
\makesavenoteenv{longtable}
\usepackage{graphicx}
\makeatletter
\def\maxwidth{\ifdim\Gin@nat@width>\linewidth\linewidth\else\Gin@nat@width\fi}
\def\maxheight{\ifdim\Gin@nat@height>\textheight\textheight\else\Gin@nat@height\fi}
\makeatother
% Scale images if necessary, so that they will not overflow the page
% margins by default, and it is still possible to overwrite the defaults
% using explicit options in \includegraphics[width, height, ...]{}
\setkeys{Gin}{width=\maxwidth,height=\maxheight,keepaspectratio}
% Set default figure placement to htbp
\makeatletter
\def\fps@figure{htbp}
\makeatother

\usepackage[noblocks]{authblk}
\renewcommand*{\Authsep}{, }
\renewcommand*{\Authand}{, }
\renewcommand*{\Authands}{, }
\renewcommand\Affilfont{\small}
\makeatletter
\@ifpackageloaded{tcolorbox}{}{\usepackage[skins,breakable]{tcolorbox}}
\@ifpackageloaded{fontawesome5}{}{\usepackage{fontawesome5}}
\definecolor{quarto-callout-color}{HTML}{909090}
\definecolor{quarto-callout-note-color}{HTML}{0758E5}
\definecolor{quarto-callout-important-color}{HTML}{CC1914}
\definecolor{quarto-callout-warning-color}{HTML}{EB9113}
\definecolor{quarto-callout-tip-color}{HTML}{00A047}
\definecolor{quarto-callout-caution-color}{HTML}{FC5300}
\definecolor{quarto-callout-color-frame}{HTML}{acacac}
\definecolor{quarto-callout-note-color-frame}{HTML}{4582ec}
\definecolor{quarto-callout-important-color-frame}{HTML}{d9534f}
\definecolor{quarto-callout-warning-color-frame}{HTML}{f0ad4e}
\definecolor{quarto-callout-tip-color-frame}{HTML}{02b875}
\definecolor{quarto-callout-caution-color-frame}{HTML}{fd7e14}
\makeatother
\makeatletter
\@ifpackageloaded{caption}{}{\usepackage{caption}}
\AtBeginDocument{%
\ifdefined\contentsname
  \renewcommand*\contentsname{Table of contents}
\else
  \newcommand\contentsname{Table of contents}
\fi
\ifdefined\listfigurename
  \renewcommand*\listfigurename{List of Figures}
\else
  \newcommand\listfigurename{List of Figures}
\fi
\ifdefined\listtablename
  \renewcommand*\listtablename{List of Tables}
\else
  \newcommand\listtablename{List of Tables}
\fi
\ifdefined\figurename
  \renewcommand*\figurename{Figure}
\else
  \newcommand\figurename{Figure}
\fi
\ifdefined\tablename
  \renewcommand*\tablename{Table}
\else
  \newcommand\tablename{Table}
\fi
}
\@ifpackageloaded{float}{}{\usepackage{float}}
\floatstyle{ruled}
\@ifundefined{c@chapter}{\newfloat{codelisting}{h}{lop}}{\newfloat{codelisting}{h}{lop}[chapter]}
\floatname{codelisting}{Listing}
\newcommand*\listoflistings{\listof{codelisting}{List of Listings}}
\makeatother
\makeatletter
\makeatother
\makeatletter
\@ifpackageloaded{caption}{}{\usepackage{caption}}
\@ifpackageloaded{subcaption}{}{\usepackage{subcaption}}
\makeatother
\ifLuaTeX
  \usepackage{selnolig}  % disable illegal ligatures
\fi
\usepackage[]{natbib}
\bibliographystyle{plainnat}
\usepackage{bookmark}

\IfFileExists{xurl.sty}{\usepackage{xurl}}{} % add URL line breaks if available
\urlstyle{same} % disable monospaced font for URLs
\hypersetup{
  pdftitle={Project Submission Guidelines},
  pdfauthor={© Prof.~Dr.~Stephan Huber},
  colorlinks=true,
  linkcolor={blue},
  filecolor={magenta},
  citecolor={magenta},
  urlcolor={blue},
  pdfcreator={LaTeX via pandoc}}

  \title{Project Submission Guidelines}
  
    \subtitle{Data Science for Business (SS 2024)}
  
    \author{© Prof.~Dr.~Stephan Huber}
    \affil{%
          Hochschule Fresenius - University of Applied Science
        }
  \affil{%
          Email: stephan.huber@hs-fresenius.de
        }
  \affil{%
          Website: https://hubchev.github.io
        }
      
  \date{May 1, 2024}
\begin{document}
\maketitle
\begin{abstract}
This paper outlines the project requirements for the Data Science for
Business course. It provides guidance for efficient progress and
success, explains the components and files required for submission, and
clarifies how a submission will be evaluated.
\end{abstract}

\renewcommand*\contentsname{Table of contents}
{
\hypersetup{linkcolor=}
\setcounter{tocdepth}{5}
\tableofcontents
}
\clearpage

\section{Project description}\label{sec-intro}

Students complete this module with a project that contains

\begin{itemize}
\tightlist
\item
  a written report (10-15 written pages per student) and
\item
  a presentation, lasting for 10-15 minutes per student with a
  subsequent discussion.
\end{itemize}

Students show that they are capable of describing the status of their
work, their approach, findings and results. The presentation and
subsequent discussion take place during the lecture period; the exact
date is set by the lecturer. Group work is permitted. In case of group
work, it must be possible to clearly define and assess each student's
individual performance on the basis of specified sections, page numbers,
or other objective criteria.

This year's project focuses on demonstrating the reproducibility of an
empirical academic paper by accurately reproducing some of its empirical
results. Students must consult with the lecturer to determine which
sections of the chosen paper to replicate. This project is an
opportunity for students to demonstrate their mastery of empirical
research methodologies using R, as well as their proficiency with
essential data science tools, including Markdown, Quarto, git, GitHub,
and BibTeX.

\section{Details about the things to
do}\label{details-about-the-things-to-do}

\subsection{Submit your preferences}\label{submit-your-preferences}

Below you find a list of papers. Your task is to reproduce the empirical
results of one of these papers. \textbf{I will assign you a paper by May
9.} You can influence my assignment by providing me with a list of your
three preferred papers. \textbf{Please send me your preferences no later
than May 8}. I will do my best to take your preferences into account.

\begin{enumerate}
\def\labelenumi{\arabic{enumi}.}
\tightlist
\item
  \citet{Bachas2024Tax}: \emph{Tax Equity in Low- and Middle-Income
  Countries}
\item
  \citet{Barrero2023Evolution}: \emph{Evolution of Work Patterns
  Post-COVID-19}
\item
  \citet{Chyn2021Neighborhoods}: \emph{Moved to Opportunity: The
  Long-Run Effects of Public Housing Demolition on Children}
\item
  \citet{Cochrane2022Fiscal}: \emph{Fiscal Theory of the Price Level}
\item
  \citet{Corrado2022Intangible}: \emph{Intangible Capital and Growth
  Strategies for Advanced Economies}
\item
  \citet{Deming2022Four}: \emph{The Growing Importance of Social Skills
  in the Labor Market}
\item
  \citet{Jack2023COVID}: \emph{COVID-19 and Educational Attainment:
  Learning Loss in the Covid Era}
\item
  \citet{Jones2021Rise}: \emph{The Rise of Innovation in China:
  Challenges and Opportunities}
\item
  \citet{Kearney2022Puzzle}: \emph{Labor Market Challenges and
  Opportunities in the Post-Pandemic Era}
\item
  \citet{Kreiner2022Danish}: \emph{Inequality in the 21st Century: A
  Danish Perspective}
\item
  \citet{Levinson2023Are}: \emph{Are Environmental Regulations Effective
  in Promoting Sustainable Development?}
\item
  \citet{Lissoni2024Migration}: \emph{Migration and Innovation: Learning
  from Patent and Inventor Data}
\item
  \citet{Marie2024Immigration}: \emph{Immigration and Crime: An
  International Perspective}
\item
  \citet{Morgan2023Economic}: \emph{Economic Considerations for Health
  Policy Post-Pandemic}
\item
  \citet{Okunogbe2024How}: \emph{How Can Lower-Income Countries Collect
  More Taxes? The Role of Technology, Tax Agents, and Politics}
\item
  \citet{Price2023What}: \emph{What Can Historically Black Colleges and
  Universities Teach about Improving Higher Education Outcomes for Black
  Students?}
\item
  \citet{Rogoff2022Emerging}: \emph{Emerging Markets and the Global
  Economy in the Post-COVID World}
\item
  \citet{Sloane2021College}: \emph{College Admissions in America:
  Challenges for Diversity and Inclusion}
\end{enumerate}

\subsection{Conduct the reproduction
study}\label{conduct-the-reproduction-study}

Your task is to replicate the paper's results using the programming
language R. Recognizing the constraints of time, fully reproducing every
statistic, table, and graph from the paper might not be feasible, and
that's perfectly acceptable.

To ensure a focused and achievable project scope, please talk with me at
least once during your study to align on what aspects are essential to
replicate or investigate further. This consultation will help us stay in
sync and clarify the priorities for your work.

I encourage you to reach out to me proactively instead of waiting for me
to initiate contact to ensure that we are fully aligned and understand
each other's expectations and progress.

\subsection{Prepare a presentation and publish it on GitHub
Pages}\label{prepare-a-presentation-and-publish-it-on-github-pages}

Create a presentation using (R) Markdown and \textbf{Quarto}, and
subsequently \textbf{publish it as a website through GitHub}. How to use
Markdown and Quarto as well as how to publish a website on GitHub is
explained \href{https://hubchev.github.io/dsbl/}{here}
\citep[see][]{Huber2024Data}.

Given the limited presentation time, prioritize key points to ensure you
stay within the given timeframe without overly promoting yourself.
Briefly describe and present the research paper, focusing more
extensively on the dataset utilized in your study. The presentation
should serve as a progress report, highlighting ongoing work rather than
concluded results.

If you encounter weaknesses or challenges in conducting your
reproduction study, the presentation is an appropriate platform to share
these. The presentation is not the occasion for showcasing success
stories. Similar to an internal business meeting, the interest lies in
understanding the hurdles you face, as this opens the door for
constructive feedback and suggestions that could help overcome these
challenges.

\subsection{Write the report}\label{write-the-report}

The report

\begin{itemize}
\tightlist
\item
  must be written with Quarto,
\item
  should contain 4000-5000 words, or approximately 15 double-spaced
  pages, and
\item
  should be published in

  \begin{itemize}
  \tightlist
  \item
    html standalone format and
  \item
    PDF format.
  \end{itemize}
\end{itemize}

Please note that this report is different from an academic paper in that
it should focus solely on documenting, discussing, and presenting your
project. Its purpose is to introduce your work to me in a way that is
similar to reports written in business settings, where you focus on
explaining what you did. Additionally, you should

\begin{itemize}
\tightlist
\item
  motivate your work and your procedure,
\item
  mention briefly obstacles you overcame,
\item
  discuss what challenges, problems and weaknesses remain, and
\item
  suggest a strategy proceeding with your work if you would have had
  more time and resources.
\end{itemize}

Please refrain from trying to impress me with a fancy layout or any
extraneous details. Your primary focus should be on effectively
communicating your current state of work to the reader. Feel free to
include anything that can help achieve this goal.

Please put some emphasize on guiding and motivating the reader. For
example, the introduction is a good place to introduce the scope and
content of the report. To ensure conciseness and clarity, please
eliminate all unnecessary repetition. Take the time to read each
sentence multiple times and ask yourself if it is concise, clear, and
coherent with what was said before and after.

I recommend writing the report as a Quarto book.
\citet{Telford2023Markdown} is a good tutorial on how to write with
Markdown and Quarto. Additionally, I recommend reading
\citet{Huber2024Data}. For guidance on creating a standalone HTML file,
refer to
\href{https://quarto.org/docs/output-formats/html-publishing.html\#standalone-html}{this
resource}.

Incorporate all R code relevant to reproducing the empirical findings
directly into your Quarto file using code chunks. Your QMD file(s) must
document the complete workflow, encompassing data import, cleaning, and
analysis. While all code should be included, it's not necessary to
display every message and output generated by the code in the PDF
document.

The outline of the paper must contain at least the following building
blocks:

\begin{itemize}
\tightlist
\item
  Title and all common personal details (name, email, \ldots).
\item
  Abstract of the paper (which highlights the content of the document).
\item
  All the R code that is necessary to replicate your results.
\item
  A section where you explain briefly how you published your
  presentation on GitHub, see Section~\ref{sec-gitgithub}.
\item
  A section where you explain briefly, how you made the pull request,
  see Section~\ref{sec-gitgithub}.
\item
  The Affidavit, see Section~\ref{sec-affidavit}.
\end{itemize}

\subsection{Make a pull request with Git and
GitHub}\label{sec-gitgithub}

As mentioned above, you should publish your presentation using GitHub
pages. Furthermore, you are required to make a pull request to my Github
repository:
\href{https://github.com/hubchev/make_a_pull_request}{make\_a\_pull\_request}.
What you should do here in detail is explained in the README of the repo
and in \citet{Huber2024Data}. Remember to reference this pull request in
your report.

\subsection{Add this affidavit to your report}\label{sec-affidavit}

\emph{Your report should contain the following \textbf{Affidavit}.
Simply, fill it out and put it at the end of your report. You can check
the box like this:}

\begin{itemize}
\tightlist
\item[$\boxtimes$]
  I checked this box
\end{itemize}

\begin{tcolorbox}[enhanced jigsaw, arc=.35mm, colback=white, opacityback=0, left=2mm, breakable, rightrule=.15mm, bottomrule=.15mm, leftrule=.75mm, toprule=.15mm, colframe=quarto-callout-tip-color-frame]

I hereby affirm that this submitted paper was authored unaided and
solely by me. Additionally, no other sources than those in the reference
list were used. Parts of this paper, including tables and figures, that
have been taken either verbatim or analogously from other works have in
each case been properly cited with regard to their origin and
authorship. This paper either in parts or in its entirety, be it in the
same or similar form, has not been submitted to any other examination
board and has not been published.

I have read the Handbook of Academic Writing by
\citet{Hildebrandt2019Handbook} and have endeavored to comply with the
guidelines and standards set forth therein.

I acknowledge that the university may use plagiarism detection software
to check my thesis. I agree to cooperate with any investigation of
suspected plagiarism and to provide any additional information or
evidence requested by the university.

The report includes:

\begin{itemize}
\tightlist
\item[$\square$]
  About 4000 words (+/- 500).
\item[$\square$]
  A title page with personal details (name, email, matriculation
  number).
\item[$\square$]
  An abstract.
\item[$\square$]
  A bibliography, created using BibTeX with APA citation style.
\item[$\square$]
  The complete R code required to reproduce the results.
\item[$\square$]
  Detailed instructions on data acquisition and importation into R.
\item[$\square$]
  An introduction to guide the reader and a conclusion summarizing the
  work and discussing potential future extensions.
\item[$\square$]
  All significant resources used in the report and R code development.
\item[$\square$]
  The filled out Affidavit.
\item[$\square$]
  A concise description of the successful use of Git and GitHub, as
  detailed here:
  \href{https://github.com/hubchev/make_a_pull_request}{make\_a\_pull\_request}.
\item[$\square$]
  A concise description of the presentation published on GitHub.
\end{itemize}

The project submission includes:

\begin{itemize}
\tightlist
\item[$\square$]
  The .qmd file(s) of the report.
\item[$\square$]
  The \_quarto.yml file of the report.
\item[$\square$]
  The .pdf file of the report.
\item[$\square$]
  The standalone .html file of the report.
\item[$\square$]
  All necessary files (not available online) to reproduce the report and
  the R code.
\item[$\square$]
  The standalone .html file of the presentation.
\end{itemize}

{[}Your Name,{]} {[}Date,{]} {[}Place{]}

\end{tcolorbox}

\subsection{Submit via ILIAS}\label{submit-via-ilias}

\begin{itemize}
\tightlist
\item
  Please consider the deadline for academic papers and written
  assessments!
\item
  Upload \textbf{one .zip file} containing the following:

  \begin{enumerate}
  \def\labelenumi{\arabic{enumi}.}
  \tightlist
  \item
    the paper as (a) .pdf and a (b) .html file.
  \item
    the .qmd file
  \item
    the presentation as .html file,
  \item
    additional files, if needed, so that I can evaluate your work.
  \end{enumerate}
\end{itemize}

\section{Evaluation}\label{evaluation}

\begin{itemize}
\item
  \emph{65 \% -- Quality and execution of the project} -- After your
  presentation, we will discuss your work in a personal meeting. The
  goal of this conversation will be that we agree on certain standards
  by which I will grade you. By this I mean that we define certain goals
  that you should achieve with your data set and your question. The goal
  is to create a transparent set of expectations on my part. So that you
  have an indication of what you need to accomplish at a minimum in
  order to pass the course.
\item
  \emph{35 \% -- Quality and execution of the presentation}
\item
  I will try to evaluate your work as objectively as possible. In
  particular, I will

  \begin{itemize}
  \tightlist
  \item
    check whether your submission is complete, or not,
  \item
    check whether your empirical work can be reproduced,
  \item
    check if all formal criteria are met,
  \item
    check for plagiarism,
  \item
    check if the replication of the paper was already done with R by
    somebody else,
  \item
    read your work and evaluate your writing skills (clarity, coherence,
    grammar, etc.),
  \item
    review and evaluate the difficulty level of your project,
  \item
    evaluate the technical level of use of the programming language R
    for your empirical goals,
  \item
    assess whether your empirical reasoning makes sense and discuss your
    remaining weaknesses,
  \item
    acknowledge your learning process.
  \end{itemize}
\end{itemize}

\section{List of paper}\label{sec-projects}

\section{Literature}\label{literature}

\renewcommand{\bibsection}{}
\bibliography{aeaweb.bib}




\end{document}
